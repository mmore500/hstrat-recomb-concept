\begin{sidewaysfigure}
  \centering
  \begin{minipage}{.7\textwidth} % adjust the width as needed

    \begin{minipage}{\textwidth}
      \centering
      \includegraphics[height=0.30\textheight]{img/copy-count-snapshot}
      \subcaption{Cartoon depiction of delayed copy count estimation mechanism, annotated over Muller plot depicting weak selection over focal allele.
      }
      % \label{fig:ne-example-replicates:bottleneck}
    \end{minipage}%

    \vspace{1cm}

    \begin{minipage}{0.5\textwidth}
      \centering
      \includegraphics[height=0.2\textheight]{notebooks/notebooks/teeplots/fit=0.0+hue=clade+multiple=stack+ngen=16+npop=100+palette=set2+viz=histplot+x=generation+ext=}
      \subcaption{Muller plot depicting no selection for focal allele, ending with smaller copy count after 16 generations.}
      % \label{fig:ne-example-replicates:selection_pressure}
    \end{minipage}%
    \begin{minipage}{0.5\textwidth}
      \centering
      \includegraphics[height=0.2\textheight]{notebooks/notebooks/teeplots/fit=1.0+hue=clade+multiple=stack+ngen=16+npop=100+palette=set2+viz=histplot+x=generation+ext=}
      \subcaption{Muller plot depicting strong selection for focal allele, with fixation occuring before 16 generations.}
      % \label{fig:ne-example-replicates:control}
    \end{minipage}

  \end{minipage}
  \hfill % Creates horizontal space. Can also use \hspace{<len>}
  \begin{minipage}{.25\textwidth} % adjust the width as needed
    \caption{
      Proposed mechanism for detecting gene-level selection via a distributed delayed copy count estimation mechanism.
      Strata deposited at generation $n$ progress through 16 generations, with copy count of one allele growing due to selection.
      On the sixteenth generation, a ``snapshot'' is performed to set a random bit on field annotated onto each descendant differentia copy.
      In subsequent recombination events, set bits are exchanged between bit fields associated with common differentia.
      Copy count at generation $n + 16$ from can then estimated from these bit fields, with high copy count being suggestive of selection.
      Note that in this example collision between set bits $i'$ and $i"$ result in an undercount.
      This mechanism is associated with ``gene-level'' instrumentation (Figure \ref{fig:annotation-types}).
    }
    \label{fig:ne-example-replicates}
  \end{minipage}

\end{sidewaysfigure}


% notebooks/notebooks/teeplots/notebook=ne-inference+replicate=0+treatment=bottleneck+viz=plot-running-estimation+x=rank+y=population-size+ext=.pdf
%
% notebooks/notebooks/teeplots/notebook=ne-inference+replicate=0+treatment=control+viz=plot-running-estimation+x=rank+y=population-size+ext=.pdf
%
% notebooks/notebooks/teeplots/notebook=ne-inference+replicate=0+treatment=range-expansion+viz=plot-running-estimation+x=rank+y=population-size+ext=.pdf
%
% notebooks/notebooks/teeplots/notebook=ne-inference+replicate=0+treatment=selection-pressure+viz=plot-running-estimation+x=rank+y=population-size+ext=.pdf
