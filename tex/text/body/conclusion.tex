\section{Conclusion} \label{sec:conclusion}

Nature operates as a fully distributed system.
Methodology developed to provide our increasingly rich account of natural history often composes a larger picture through biomaterial samplings and best-effort field observations.
It seems prudent, therefore, to look to biology not only for inspiration in engineering algorithms that employ ecology and evolution, but also in devising methodology to observe them at scale.

To this end, this work has explored reconstruction-based approaches to phylogenetic analysis.
Proposed instruments collect ``gene-'' and ``species-level'' information from fully-distributed EC systems that employ sexual recombination.
This is achieved through extension of hereditary stratigraphy genome annotations originally designed for completely asexual populations.
Experiments validated capability to detect aspects of genealogical history, demographic history, and selection dynamics, all without centralized tracking.

The ultimate aim of this project is to provision infrastructure for phylogenetic observation to any pertinent digital evolution system, with a special eye to large-scale distributed processing.
To this end, open source, plug-and-play software implementation of hereditary stratigraphy algorithms and data structures are core priority.
Ample opportunity exists for collaboration to tailor hereditary stratigraphy techniques and software to applications across evolution systems, programming languages, and underlying hardware.
