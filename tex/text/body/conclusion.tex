\section{Conclusion} \label{sec:conclusion}

Nature operates as a fully distributed system.
Implicitly, therefore, methodology developed to provide our increasingly rich account of natural history often compose a larger picture through samplings of biomaterials and best-effort field observations.
It seems prudent, therefore, to look to biology not only for inspiration in engineering ecological and evolutionary dynamics within EC systems, but also in devising methodology to measure them at scale.
This work has introduced schemes to collect ``gene-'' and ``species-level'' information in fully distributed EC systems with sexual recombination.
This is achieved through extension of existing hereditary stratigraphy genome annotation originally designed for populations with asexual reproduction.
Experiments confirmed proposed genome annotation schemes can reveal aspects of genealogical history, demographic history, and selection dynamics without centralized tracking.

The ultimate aim of this project is to provision infrastructure for phylogenetic observation to any prospective large-scale digital evolution system.
To this end, open source, plug-and-play software implementation of hereditary stratigraphy algorithms and data structures are core priority.
Ample opportunity exists for collaborations to tailor application of hereditary stratigraphy to particular evolution systems, programming languages, and underlying hardware.
