\section{Results and Discussion} \label{sec:results}

This section reports validation experiments for debuted genealogical, effective population size, and positive selection inference techniques (Sections \ref{sec:genealogical-inference}, \ref{sec:population-size-inference}. and \ref{sec:selection-inference}).

\subsection{Genealogical Inference}
\begin{figure}
  \begin{subfigure}{.45\linewidth}
    \centering
    \includegraphics[width=1.05\linewidth]{notebooks/notebooks/teeplots/max_leaves=20+notebook=species-inference+replicate=0+treatment=bag+type=distilled-reference+viz=draw-biopython-tree+ext=}
    \caption{Bag reference}
    \label{fig:species-example-replicates:bag-reference}
  \end{subfigure}
  \,
  \begin{subfigure}{.45\linewidth}
    \centering
    \includegraphics[width=1.05\linewidth]{notebooks/notebooks/teeplots/max_leaves=20+notebook=species-inference+replicate=0+treatment=bag+type=reconstruction+viz=draw-biopython-tree+ext=}
    \caption{Bag reconstruction}
    \label{fig:species-example-replicates:bag-reconstruction}
  \end{subfigure}

  \begin{subfigure}{.45\linewidth}
    \centering
    \includegraphics[width=1.05\linewidth]{notebooks/notebooks/teeplots/max_leaves=20+notebook=species-inference+replicate=0+treatment=allopatry+type=distilled-reference+viz=draw-biopython-tree+ext=}
    \caption{Allopatry reference}
    \label{fig:species-example-replicates:allopatry-reference}
  \end{subfigure}
  \,
  \begin{subfigure}{.45\linewidth}
    \centering
    \includegraphics[width=1.05\linewidth]{notebooks/notebooks/teeplots/max_leaves=20+notebook=species-inference+replicate=0+treatment=allopatry+type=reconstruction+viz=draw-biopython-tree+ext=}
    \caption{Allopatry reconstruction}
    \label{fig:species-example-replicates:allopatry-reconstruction}
  \end{subfigure}

  \begin{subfigure}{.45\linewidth}
    \centering
    \includegraphics[width=1.05\linewidth]{notebooks/notebooks/teeplots/max_leaves=20+notebook=species-inference+replicate=0+treatment=ring+type=distilled-reference+viz=draw-biopython-tree+ext=}
    \caption{Ring reference}
    \label{fig:species-example-replicates:ring-reference}
  \end{subfigure}
  \,
  \begin{subfigure}{.45\linewidth}
    \centering
    \includegraphics[width=1.05\linewidth]{notebooks/notebooks/teeplots/max_leaves=20+notebook=species-inference+replicate=0+treatment=ring+type=reconstruction+viz=draw-biopython-tree+ext=}
    \caption{Ring reconstruction}
    \label{fig:species-example-replicates:ring-reconstruction}
  \end{subfigure}
  \caption{Comparison of reference phylogenies (top) and reconstructed phylogenies (bottom) for example replicates of each experimental treatment: bag (well-mixed), allopatry (population split at generation 100 with a secondary split on one branch at generation 150), and ring (ten island subpopulations with some migration).
  Phlogenies are downsampled from 100 to 20 tips for legibility.
  Extant organism ID's are annotated on tips.
  Taxon color coding is consistent between reference and reconstruction to facilitate comparison.
  Branch length on $x$ axis given in generations.
  }
  \label{fig:species-example-replicates}

\end{figure}
%
% notebooks/notebooks/teeplots/max_leaves=20+notebook=species-inference+replicate=0+treatment=allopatry+type=distilled-reference+viz=draw-biopython-tree+ext=.pdf
%
% notebooks/notebooks/teeplots/max_leaves=20+notebook=species-inference+replicate=0+treatment=allopatry+type=reconstruction+viz=draw-biopython-tree+ext=.pdf
%
% notebooks/notebooks/teeplots/max_leaves=20+notebook=species-inference+replicate=0+treatment=bag+type=distilled-reference+viz=draw-biopython-tree+ext=.pdf
%
% notebooks/notebooks/teeplots/max_leaves=20+notebook=species-inference+replicate=0+treatment=bag+type=reconstruction+viz=draw-biopython-tree+ext=.pdf
%
% notebooks/notebooks/teeplots/max_leaves=20+notebook=species-inference+replicate=0+treatment=ring+type=distilled-reference+viz=draw-biopython-tree+ext=.pdf
%
% notebooks/notebooks/teeplots/max_leaves=20+notebook=species-inference+replicate=0+treatment=ring+type=reconstruction+viz=draw-biopython-tree+ext=.pdf


Figure \ref{fig:species-example-replicates} compares phylogenetic trees reconstructed from species-level instrumentation to corresponding references extracted from perfectly-tracked sexual pedigrees.
For treatments with meaningful phylogentic structure --- the ``allopatry'' and ``ring'' treatments --- phylogenetic reconstruction largely succeeded in recovering the historical relationships between subpopulations.
In fact, for the ``allopatry'' treatment, inner node time points appear to more closely track the true generational time frames of speciation events (at generation 100 and 150) than the UPGMA-based pedigree distillation.

Supplemental Figure \ref{fig:species-reconstruction-error} shows distributions of reconstruction error for each treatment.
Across all three treatments, all ten replicate reconstructions exhibited quartet distance from reference strictly fell below 0.66, the expected for comparison of random trees \citep{smith2020information}.
So, significant phylogenetic information was recovered in all three cases (exact binomial test, $p < 0.01$).

However, as expected, reconstruction quality on the bag population structure was marginal due to the lack of meaningful phylogenetic information available to reconstruct.
Performance on the ring and allopatry treatments was stronger, achieving quartet distances between reconstruction and reference of around 0.3 in the typical case.
However, inclusion of nebulous phylogenetic structure within the reference phylogeny obscures the true magnitude of meaningful reconstruction error.

\subsection{Population Size Inference.}
Supplemental Figure \ref{fig:ne-example-replicates} shows ten-sample rolling estimates of population size for one replicate from each surveyed treatment.
All effective population estimates respond to underlying demographic changes, although the response to selection pressure relaxation appears weaker than responses to changes in population size.
Substantial estimate volatility appears across all cases.

\begin{figure}
  \centering
  \includegraphics[width=0.8\textwidth]{notebooks/notebooks/teeplots/hue=generation+viz=boxplot-popsize+x=treatment+y=population-size-estimate+ext=}
  \caption{
  Distributions of 10-sample MLE population size estimates by treatment across three time points.
  See Section \ref{sec:population-size-inference-experiments} for population size and selection pressure manipulations performed for each treatment.
  Notches indicate bootstrapped 95\% confidence intervals.}
  \label{fig:ne-estimate-distributions}
\end{figure}

% notebooks/notebooks/teeplots/hue=generation+viz=boxplot-popsize+x=treatment+y=population-size-estimate+ext=.pdf


Figure \ref{fig:ne-estimate-distributions} summarizes the distribution of effective population size estimates across replicates at three time points spread across the beginning, middle, and end of evolutionary runs.
Estimates differ across time points within all non-control treatments, more rigorously confirming estimator sensitivity.
For the bottleneck and selection pressure treatments, which involve reversion to initial conditions, estimate distributions at the first and last time points are comparable, as expected.

Supplemental Figure \ref{fig:ne-detection-matrix} summarizes the detectability of underling $N_e$ changes.
Detection was performed using MLE confidnece interval comparison between rolling population size estimates at different time points
No false positives differences in effective population size are detected.
Most true changes in effective population size are detected in at least nine out of ten replicates, except for the selection pressure treatment and for one segment of the range expansion treatment.
Supplemental Figure \ref{fig:ne-detection-matrix} also shows Mann-Whitney comparison of population size estimates using a larger sample size of 30 observations, which was more consistently sensitive.

\subsection{Positive Selection Inference}
\begin{sidewaysfigure}
  \centering
  \begin{minipage}{.7\textwidth} % adjust the width as needed

    \begin{minipage}{\textwidth}
      \centering
      \includegraphics[width=\textwidth]{notebooks/notebooks/teeplots/col=fitness-advantage+viz=facet-lineplot-twiny+x=generation+y1=prevalence+y2=annotation+ext=}
      \subcaption{Cross-replicate aggregate, shaded bands are 95\% percentile intervals}
      \label{fig:selection-example-replicates:aggregate}
    \end{minipage}

    \vspace{1cm}

    \centering
    \begin{minipage}{0.32\textwidth}
      \centering
      \includegraphics[width=\textwidth]{notebooks/notebooks/teeplots/fitness-advantage=0.0+notebook=gene-selection-inference+replicate=0+viz=plot-sweep-and-annotations+ext=}
      \subcaption{Example replicate with fitness advantage 0.0}
      \label{fig:selection-example-replicates:fit-0-0}
    \end{minipage}
    \begin{minipage}{0.32\textwidth}
      \centering
      \includegraphics[width=\textwidth]{notebooks/notebooks/teeplots/fitness-advantage=0.1+notebook=gene-selection-inference+replicate=0+viz=plot-sweep-and-annotations+ext=}
      \subcaption{Example replicate with fitness advantage 0.1}
      \label{fig:selection-example-replicates:fit-0-1}
    \end{minipage}
    \begin{minipage}{0.32\textwidth}
      \centering
      \includegraphics[width=\textwidth]{notebooks/notebooks/teeplots/fitness-advantage=1.0+notebook=gene-selection-inference+replicate=0+viz=plot-sweep-and-annotations+ext=}
      \subcaption{Example replicate with fitness advantage 1.0}
      \label{fig:selection-example-replicates:fit-1-0}
    \end{minipage}%


  \end{minipage}
  \hfill % Creates horizontal space. Can also use \hspace{<len>}
  \begin{minipage}{.25\textwidth} % adjust the width as needed
    \caption{
    Gene prevalence trajectories (shown in red) and 16-generation rolling sums of gene prevalence annotation bit counts (shown in blue) across generations by selection strength treatment.
    Top row summarizes distribution across replicates.
    Bottom row shows an example replicate of each treatment.
    Fitness advantage 0.0 inferred no selective benefit, so all selection detections on this treatment are false positives.
    Fitness advantage 0.1 experienced relatively weak selection and fitness advantage 1.0 experienced strong selection.
    Spikes of high gene prevalence annotation bit count (blue) are indicative of selective dynamics.
    Selection is detected for a replicate if any 16-generation rolling sum of gene prevalence annotation bit count (Section \ref{sec:dist-gene-prevalence-est}) exceeds the threshold.
    Note that $y$ axis scaling differs among bottom-row graphs.
    }
    \label{fig:selection-example-replicates}
  \end{minipage}

\end{sidewaysfigure}

Gene selection experiments introduced novel alleles with fitness advantages of 1.0 (strong selection), 0.1 (weaker selection), and 0.0 (no selection --- control).
Figure \ref{fig:selection-example-replicates} jointly plots underlying gene copy count and delayed copy count estimates extracted from a randomly-sampled population member.
For the strong selection treatment, allele frequency fixes rapidly and induces a sharp spike in the delayed copy count.
For the weaker selection treatment, allele frequency grows somewhat less rapidly.
A delayed copy count spike is apparent, but of smaller magnitude and --- corresponding to variation in underlying sweep timing --- occuring with variable delay.

The sensitivity and specificity of positive gene selection detection was evaluated in terms of false-positive (i.e., detection selection on control replicates) and false-negative rates (i.e., non-detection of selection on fitness-advantaged replicates) across a range of detection threshold values.
Supplemental Figure \ref{fig:selection-sensitivity-specificity} plots detectoun outcomes across a range of threshold values.
Strong selection events can be unambiguously distinguished from neutral events, as well as from weaker selection events.
Weaker selection and neutral events were not entirely separable.
Setting the detection threshold at sum count 200 misidentified one neutral event and one weak positive event --- corresponding to a 10\% false-positive and 10\% false-negative rate.
