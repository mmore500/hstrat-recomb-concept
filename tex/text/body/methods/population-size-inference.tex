\subsection{Population Size Inference}
\label{sec:population-size-inference}

This section introduces the mechanistic principle behind distributed population size estimation, reports statistical formulations derived to this end, then describes experiments that test the proposed inference technique.

\textbf{Inference Principle.}
Recall that the proposed gene drive mechanism (Section \ref{sec:genome-instrumentation}) fixes the maximum of population size $n$ fingerprint values, each drawn from a uniform distribution.
Under this gene drive mechanism, fixed fingerprint magnitude reveals information about population size: larger populations tend to fix greater fingerprints (as illustrated in Supplementary Figure \ref{fig:beta-explain}).
Scaling integer fingerprint values to fall between 0 and 1, fixed fingerprint magnitude turns out to follow beta distribution $\beta(n, 1)$ \citep{gentle2009computational}.

Directly analogous techniques to estimate population size have also arisen in decentralized, anonymous network engineering \citep{varagnolo2010distributed,hakan2012distributed}.
In these schemes, nodes independently draw a random vector of numerical values from a known distribution.
Values are exchanged through an aggregating function (e.g., minimum, maximum, etc.), ultimately resulting in consensus values fixed within the network.
Each node can then independently infer probabilistic information about the larger network, in a manner that is generally consistent across nodes.

\textbf{Population Size Estimator Statistics.}
Statistical details for population size inference follow, some of which are, to best knowledge, not yet reported.
Natural log is used throughout.
Section \ref{sec:software-data} links to full derivations.

\textit{Maximum Likelihood Estimator (MLE).}
The maximum likelihood estimator for population size given $k$ independent observations of unit-scaled fixed-fingerprint values $x_i$ is
\begin{footnotesize}
\begin{align} \label{eqn:popsize_mle}
\hat{n}_\mathrm{mle} = -k/\textstyle\sum_{i=1}^k \log( x_i ).
\end{align}
\end{footnotesize}

With true population size $n$, mean square error of the estimator is $\mathrm{MSE}(\hat{n}_\mathrm{mle}) = n^2 (k^{2}+ k-2) / [(k-1)^{2}(k-2)]$.
Expected value follows as $\mathrm{E}(\hat{n}_\mathrm{mle}) = nk/(k-1)$ for $k>1$.
Subtracting this value from $\hat{n}_\mathrm{mle}$ yields a mean-unbiased population size estimator.
These MLE results were also derived in \citep{varagnolo2010distributed}.

\textit{Confidence Interval.}
Confidence intervals are useful to interpretation of population size estimates.
Formulations derived from the maximum likelihood estimator are provided below.

For a single observation of unit-scaled fixed-fingerprint magnitudes $\hat{x}$, the population size $n$ can be estimated with $c\%$ confidence to fall between lower bound $\log[(50+0.5c)/100] / \log\hat{x}$ and upper bound $\log[(50-0.5c)/100] / \log\hat{x}$.
At this low observational power, the 95\% confidence interval spans a 145-fold order of magnitude and a 99\% confidence interval spans a 1057-fold order of magnitude.

For $k$ observations of unit-scaled fixed-fingerprint magnitudes $\hat{x}_i$, population size can be estimated with $c\%$ confidence to fall within the interval $(\hat{n}_\mathrm{lb}, \, \hat{n}_\mathrm{ub})$, computed as numerical solutions of

\begin{footnotesize}
\begin{align}
0
= 2\Gamma\Big(k, -\hat{n}_\mathrm{lb}\sum_{i=1}^k \log\hat{x}_i\Big) - \Gamma(k)\frac{100+c}{100} \text{ and }
0
= 2\Gamma\Big(k, -\hat{n}_\mathrm{ub}\sum_{i=1}^k \log\hat{x}_i\Big) - \Gamma(k)\frac{100-c}{100}.  \label{eqn:popsize_mle_ci}
\end{align}
\end{footnotesize}

Here, $\Gamma$ is the complete gamma function.
Four independent observations provide a 95\% confidence interval spanning 8-fold magnitude or a 99\% confidence interval spanning a 16-fold magnitude.
Nine independent observations are sufficient for a 95\% confidence interval spanning a factor spanning 4-fold magnitude or a 99\% confidence interval spanning a factor of 6-fold magnitude.
Thirty-three independent observations are sufficient for a 95\% confidence interval spanning 2-fold magnitude or a 99\% confidence interval spanning 2.5-fold magnitude.
Because $\sum_{i=1}^k \log\hat{x}_i \propto \hat{n}_\mathrm{mle}^{-1}$, confidence bound width can be shown to scale as a constant factor of population size $n$.

\textit{Median-unbiased Estimator.}
Evaluating either confidence interval formula with $c = 50$ derives a median-unbiased estimator.

\textit{Credible Intervals.}
Assuming a uniform prior over population size, credibility contained within a factor $f$ of the maximum likelihood estimate $\hat{n}_\mathrm{mle}$ can be calculated as $\gamma(k + 1, f k)/k! - \gamma(k + 1, k/f)/k!$.
Here, $\gamma$ is the lower incomplete gamma function.
By form, this credibility remains constant across population sizes $n$.
Credibility interval width scales similarly with sample size as discussed above for confidence intervals.

\textit{Rolling Estimation.}
Experiments reported here compute a population size estimate and confidence intervals from a simple rolling collection of ten preceding fixed-gene magnitudes.
Supplemental Section \ref{sec:population-size-inference-example} walks through an example of rolling population size inference.
More sophisticated regularizations have been proposed to estimate dynamically-changing population sizes from time series data \citep{hakan2012distributed}.

\textbf{Validation Experiment.}
This experiment tests ability of the population size estimation process to detect differences between populations and to detect changes within a population over time.
Four treatments were evaluated with ten independent replicates performed per treatment.

\textit{Bottleneck treatment.}
Simulated population crash and rebound.
Population size was kept at 100 for 67 generations, reduced an order of magnitude to 10 for 66 generations, and then returned to 100 for another 67 generations.

\textit{Range expansion.}
Simulated gradual growth in population size.
Population size was initiated at at 10 for 67 generations, then increased linearly for 66 generations to 142 at generation 133, and then maintained at 142 for another 67 generations.

\textit{Selection-pressure.}
Modulated selection intensity.
This affects effective population size by changing the number of population members eliminated without contributing to the gene pool.
High selection pressure was applied for 67 generations (tournament size 8). Then, selection pressure was eliminated for 66 generations (tournament size 1).
Finally, high selection pressure was reinstated for the last 67 generations (tournament size 8).

\textit{Control treatment.} Constant population size 100 for 200 generations.
