\subsection{Population Size Inference}
\label{sec:population-size-inference}

This section introduces the mechanistic principle behind distributed population size estimation, reports statistical formulations derived to perform distributed population size estimation, illustrates an example of inference scenario, then describe sexperiments to test the proposed inference technique.

\textbf{Inference Principle.}
Recall that the proposed gene drive mechanism (Section \ref{sec:genome-instrumentation}) fixes the maximum of population size $n$ fingerprint values, each drawn from a uniform distribution.
Under this gene drive mechanism, fixed fingerprint magnitude reveals information about population size.
As illustrated in Supplementary Figure \ref{fig:beta-explain}, larger populations tend to fix greater fingerprints..
Scaling fingerprint values to between 0 and 1, fixed fingerprint magnitude turns out to follow beta distribution $\beta(n, 1)$ \citep{gentle2009computational}.

Directly analogous techniques to estimate population size have also arisen in the context of decentralized, anonymous network engineering \citep{varagnolo2010distributed,hakan2012distributed}.
In these schemes, nodes independently draw a random vector of numerical values from a known distribution.
Values are exchanged through an aggregating function like (e.g., minimum, maximum, etc.), ultimately resulting in a consensus value fixing within the network.
Each node can then independently infer probabilistic information about the larger network, in a manner that is generally consistent across nodes.

\textbf{Population Size Estimator Statistics.}
Statistical details necessary to work with the unit-scaled fixed-fingerprint-magnitude inference method follow, some of which are, to best knowledge, not yet reported.
Section \ref{sec:software-data} links to full derivations.

\textit{Maximum Likelihood Estimator.}
The maximum likelihood estimator for population size given $k$ independent observations of unit-scaled fixed-fingerprint value $x_i$ is
\begin{footnotesize}
\begin{align} \label{eqn:popsize_mle}
\hat{n}_\mathrm{mle} = -k/\textstyle\sum_{i=1}^k \log( x_i ).
\end{align}
\end{footnotesize}

With true population size $n$, mean square error of the maximum likelihood estimator for population size is $\mathrm{MSE}(\hat{n}_\mathrm{mle}) = (n^2 (k^{2}+ k-2) / [(k-1)^{2}(k-2)]$.
Expected value for the maximum likelihood population size estimator $\hat{n}_\mathrm{mle}$ follows as $\mathrm{E}(\hat{n}_\mathrm{mle}) = nk/(k-1)$ for $k>1$.
This value can be subtracted from $\hat{n}_\mathrm{mle}$ to yield a mean-unbiased population size estimator.
Maximum likelihood estimator results were also derived in \citep{varagnolo2010distributed}.

\textit{Confidence Interval.}
Confidence intervals are useful to interpretation of population size estimates.
Formulations derived from the maximum likelihood estimator are provided below.

For a single observation of unit-scaled fixed fingerprint magnitude $\hat{x}$, the population size $n$ can be estimated with $c\%$ confidence to fall between lower bound $\log[(50+0.5c)/100] / \log\hat{x}$ and upper bound $\log[(50-0.5c)/100] / \log\hat{x}$.
At this low observational power, the 95\% confidence interval spans a 145-fold order of magnitude and a 99\% confidence interval spans a 1057-fold order of magnitude.

For $k$ observations of unit-scaled fixed gene magnitude $\hat{x}_i$, population size can be estimated with $c\%$ confidence to fall within the interval $(\hat{n}_\mathrm{lb}, \, \hat{n}_\mathrm{ub})$, computed by numerical solutions of

\begin{footnotesize}
\begin{align}
0
= 2\Gamma\Big(k, -\hat{n}_\mathrm{lb}\sum_{i=1}^k \log\hat{x}_i\Big) - \Gamma(k)\frac{100+c}{100} \text{ and }
0
= 2\Gamma\Big(k, -\hat{n}_\mathrm{ub}\sum_{i=1}^k \log\hat{x}_i\Big) - \Gamma(k)\frac{100-c}{100}.  \label{eqn:popsize_mle_ci}
\end{align}
\end{footnotesize}

Here, $\Gamma$ is the complete gamma function.
Four independent observations are sufficient to provide a 95\% confidence interval spanning 8-fold magnitude or a 99\% confidence interval spanning a 16-fold magnitude.
Nine independent observations are sufficient for a 95\% confidence interval spanning a factor spanning 4-fold magnitude or a 99\% confidence interval spanning a factor of 6-fold magnitude.
33 independent observations are sufficient for a 95\% confidence interval spanning 2-fold magnitude or a 99\% confidence interval spanning 2.5-fold magnitude.
Because $\sum_{i=1}^k \log\hat{x}_i \propto \hat{n}_\mathrm{mle}^{-1}$, confidence bound width can be shown to scale as a constant factor of population size $n$.

\textit{Median-unbiased Estimator.}
A median-unbiased estimator for population size can be trivially derived as the numerical solution to either confidence interval formula with $c = 50$.

\textit{Credible Intervals.}
Computation of credible intervals facilitates Bayesian inference.
Assuming a uniform prior over population size,
the credibility contained within a factor $f$ of the maximum likelihood estimate $\hat{n}_\mathrm{mle}$ can be calculated as $\gamma(k + 1, f k)/k! - \gamma(k + 1, k/f)/k!$.
Here, $\gamma$ is the lower incomplete gamma function.
By form, this credibility remains constant across population sizes $n$.
Credibility interval width scales similarly with sample size as discussed above for confidence intervals.

\textit{Rolling Estimation.}
Experiments reported here compute a population size estimate and confidence intervals from a simple rolling aggregate of ten preceding fixed-gene magnitudes.
However, more sophisticated regularizations have been proposed to consolidate time series estimates of dynamically-changing population sizes \citep{hakan2012distributed}.
Supplemental Section \ref{sec:population-size-inference-example} walks through an example of rolling population size inference.

\textbf{Validation Experiment.}
This experiment set out to test ability of the population size estimation process to detect population size differences between populations and to detect population size changes within a population over time.
Four different treatment conditions were compared: bottleneck, range expansion, selection pressure, and control.
Ten independent replicates were performed for each treatment.

\textit{Bottleneck treatment.}
Simulated a population crash and rebound.
The population size was kept at 100 for 67 generations, reduced an order of magnitude to 10 for 66 generations, and then returned to 100 for another 67 generations.

\textit{Range expansion.}
Simulated gradual population size expansion.
The population size was initiated at at 10 for 67 generations, then increased linearly for 66 generations to 142 at generation 133, and then maintained at 142 for another 67 generations.

\textit{Selection-pressure treatment.}
Modified the selection intensity during the evolutionary process.
This reduced the effective population size by increasing the number of population members eliminated without contributing to the next generation. gene pool.
High selection pressure was applied for 67 generations (tournament size 8). Then, selection pressure was eliminated for 66 generations (tournament size 1).
Finally, high selection pressure (tournament size 8) was reinstated for the last 67 generations.

\textit{Control treatment.} Run with a constant population size of 100 for 200 generations.
