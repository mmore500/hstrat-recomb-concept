\subsection{Genealogical Inference}
\label{sec:genealogical-inference}

This section describe the three treatments compared to test the proposed genealogical inference mechanisms then lays out the procedure used to convert recorded sexual pedigrees to phylogenetic trees that reconstruction quality could be evaluated against.

\textbf{Validation Experiment.}
This experiment tested the quality of genealogical history recovered from species-level hereditary stratigraph annotation.
Phylogenetic reconstruction quality was tested over three evolutionary treatments.
The first, ``allopatry,'' covered full speciation through introduction of a strict reproductive barrier among subpopulations.
The second, ``ring,'' covered partial phylogenetic structure over subpopulations linked through small amounts of migration.
A third --- as a control --- was designed to lack meaningful phylogenetic structure at all due to well-mixed interbreeding ensuring close relatedness between all population members.
Ten independent replicates were performed for each treatment.

The experiment was performed on the one-max problem domain described in Section \ref{sec:one-max}.
After the 200th and final generation, the species-level annotations extracted from each of the extant organisms.
Phylogenetic structure was reconstructed using from these annotations using the agglomerative trie-based reconstruction techniques developed in \citep{moreno2023toward}.
(Essentially, the phylogeny is constructed so that organisms track the lineage they share the most consecutive ``fingerprint'' differentia with and then branch out at the point of the first disparity in differentia value.

To evaluate reconstruction quality, inferred phylogenies were compared to references extracted directly from underlying sexual pedigree record of the simulation using the MRCA-based UPGMA methods described in Section \ref{sec:phylogeny-extraction}.
Disagreement between reconstruction and reference was quantified using the quartet tree distance metric \citep{estabrook1985comparison}.

Additionally, inferred phylogenies were visualized to confirm whether reconstructions recovered the major features of treatments' evolutionary histories.

\textit{Allopatry Treatment.}
This treatment simulates 100 generations of well-mixed sympatric evolution.
At generation 100, the population is divided into two 50-member subpopulations.
These subpopulations evolve independently with no migration for 50 generations.
Then, at generation 150, the first subpopulation is split again into five 10-member subpopulations.
The remaining 50-member subpopulation and the five new 10-member subpopulations then evolve independently with no migration for a further 50 generations.
Phylogenetic reconstruction from this treatment should ideally recover a binary branching at generation 100 followed by a secondary quintuple-branching along one lineage at generation 150.

\textit{Ring Treatment.}
This treatment splits the population into ten distinct subpopulations (islands), each of which evolved independently.
The subpopulations were arranged in a ring topology, and one individual migrated between adjacent populations per generation happens once per generation.

\textit{Bag Treatment.}
This treatment selects and recombines individuals uniformly from across the entire population.
As such, all individuals extant at simulation completion are closely related so no meaningful phylogenetic structure exists to be detected.

\textbf{Phylogeny Extraction from Pedigree Records.}
In order to provide a baseline reference to evaluate annotation-based phylogenetic reconstructions against, phylogenetic relationships between extant organisms (i.e., an asexual tree describing ``relatedness'') were extracted from sexual pedigrees (i.e., a reticulated directed acyclic graph describing ancestry).

Such conversion has fundamental limitations: in well-mixed populations of modest size, structured differences in phylogenetic relatedness do not meaningfully exist.
However, speciation and spatial structure can introduce meaningful aspects of phylogenetic structure.

A naive technique was used to distill phylogenetic relationships from pedigree data.
Most Recent Common Ancestors (MRCA) were computed pairwise from the pedigree data to construct a distance matrix among extant individuals.
Unweighted Pair Group Method with Arithmetic mean (UPGMA) reconstruction based on this distance matrix yielded inferred phylogenetic structure \citep{sokal1958university}.
Finally, corrections to branch lengths were performed to accurately position the terminal nodes (i.e., extant organisms) at their known generational depths.

No effort was made to cluster extant organisms into taxa based on a relatedness threshold, so reconstructions contained non-informative branch structure among closely-related individuals.
