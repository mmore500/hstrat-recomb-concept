\section{Methods} \label{sec:methods}

Three independent experiments were performed to test alternate inference techniques over sexual populations: genealogical inference, population size inference, and positive selection inference.

Section \ref{sec:one-max}

Section \ref{sec:instrumentaition}

Section \ref{sec:phylogeny-extraction}

Section \ref{sec:gene-drive}

Section \label{sec:collision-probability}

Section \label{sec:genealogical-inference}

Section \label{sec:population-size-inference-stats}

Section \label{sec:example-ne-infernce}

Section \label{sec:population-size-inference-experiments}

Section \label{sec:dist-gene-prevalence-est}

Section \label{sec:positive-selection-inference-experiment}

Finally, Section \label{sec:software-data}

\subsection{One-Max Task}
\label{sec:one-max}

We used the one-max task for experiments to evaluate population size inference and genealogical inference.
This tasks selects over fixed-length bit string individuals for the fraction of 1's set.
Each individual was represented as a binary string of length 100.
Fitness was evaluated as the sum of the bits in the binary string, preferring individuals with more 1s over individuals with fewer.
Population size was 200 individuals, initialized randomly with sites set to 1 with probability 0.5.
In treatments where population size decreased during an evolutionary run, excess individuals were eliminated randomly.
In treatments where population size increased during an evolutionary run,
new population slots were filled through selection over the existing population.

Selection was performed using the tournament selection method.
Default tournament size was 2.
Tournament sizes 1 and 8 were also set in some treatments.
Tournaments were performed synchronously, with all parents selected before turnover of the entire population.
Evolutionary runs lasted 100 generations.

Two-point crossover (mating) and bit-flip mutation with per-bit probability 0.05 were applied to all organisms at all generations prior to selection.

Operator choice and parameter selections were based on one-max example code from the Distributed Evolutionary Algorithms in Python package \citep{fortin2012deap}.

\subsection{Hereditary Stratigraphy Instrumentation}
\label{sec:instrumentation}

Proposed methods for genealogical and evolutionary inference over distributed sexual populations recently-developed ``hereditary stratigraphy'' technique, originally developed to facilitate phylogenetic inference over asexual populations \citep{moreno2022hstrat}.
This approach applies heritable markers to individual digital genomes to facilitate post-hoc reconstruction of phylogenetic history.
Each time a marker is inherited, a fresh random packet of data --- a ``fingerprint'' --- is generated and appended to the inherited marker.
To infer phylogenetic history, fingerprints from the markers of extant organisms are aligned and compared.
Identical fingerprints within the record signifies (likely) shared ancestry between organisms.
Conversely, the presence of a discrepancy in fingerprints definitively evidences divergence in ancestry at the generation those fingerprints were generated.

Without additional mechanisms, genome marker size would grow linearly as generations elapse and quickly become unwieldy.
Abatement of such instrumentation bloat necessitated development of a library of strategies for on-line pruning of fingerprints within markers.
Discarding fingerprints decreases annotation size at the cost of sparsifying time points at which  ancestral similarity or differentiation is observable.
Trade-offs between annotation size and estimation accuracy can therefore be tuned through fingerprint retention strategy.
For instance, resolution may be maintained over time points as a fixed proportion of generations elapsed since those time points at the cost of logarithmic growth in annotation size.
Or, another possibility, annotation size can be maintained below a constant bound at the cost of linear growth in uncertainty as generations elapse.

Fingerprint pruning was not performed in the reported experiments in order to simplify experimental setup and analysis.
This means that a complete annotation record with fingerprints corresponding to every generation was maintained.
Population-size and gene-selection inference mechanisms introduced here are in principle compatible with fingerprint pruning.
However, direct exploration of BEST EFFECTIVE PRACTICES with is left to future work.
In particular, to facilitate effective application of hereditary stratigraphy specifically to sexual populations, it will be critical to determine the most effective distribution of fingerprint retention (e.g., uniform across generational time versus recency-proportional favoring more recent events) and the levels of reconstruction accuracy pertinent to characterization of evolutionary history and dynamics (a\citep{moreno2023toward}.

This work uses 64 bit fingerprints, which collide with probability $1/2^{64} \approx 5 \times 10^{-20}$.
At population size 100 over 200 generations, as in the genealogical inference and control condition of the population size inference experiments, the probability of any collision is $< 2 \times 10^{-15}$.
At population size 200 over 400 generations, as in the gene selection inference experiments, the probability of any collision is $< 5 \times 10^{-15}$.

As originally devised for asexual populations, hereditary stratigraphic markers followed a one-to-one relationship to genomes.
Here, we explore two alternate schemes designed for instrumentation of sexual populations: gene instrumentation and species instrumentation.

The former treats individual genes as asexual subcomponents of sexual lineages, instrumenting genes individually using the original hereditary stratigraphy methodology.
This strategy involves (potentially) several independent hereditary stratigraph instruments per organism.
In, along the lines of ``gene tree'' analsyes in traditional phylogenetics \citep{avise1989gene}.

In cases where digital chromosomes comprise a relatively small number of atomic genes, it may make sense to instrument every gene independently.
However, in applications with very fine atomic genes (e.g., individual GP instructions) or no atomic genes (e.g., GA floating point values subjected to interpolation during crossover) it may make sense to instrument a representative subset of genes or introduce ``dummy'' genes associated with certain chromosomal positions.

The latter \citep{alphey2020standardizing, price2020resistance}, and has no direct analogy in typical phylogenetic analyses.

Species instrumentation employed the recombination and gene drive mechanism described in Section \ref{sec:gene-drive}.
Species hereditary stratigraph instrumentation was employed for the genealogical inference and population size inference experiments.
Gene instrumentation employed the distributed copy count estimation mechanism described in Section \ref{sec:copy-count}.
Gene hereditary stratigraph instrumentation was employed for the positive selection inference experiment.

\subsection{Phylogeny Extraction from Pedigree Records}
\label{sec:phylogeny-extraction}

In order to provide a baseline reference to evaluate annotation-based phylogenetic reconstructions against, phylogenetic relationships between extant organisms (i.e., an asexual tree describing ``relatedness'') were extracted from sexual pedigrees (i.e., a reticulated directed acyclic graph describing ancestry).

Such conversion has fundamental limitations: in well-mixed populations of modest size, structured differences in phylogenetic relatedness do not meaningfully exist.
However, speciation and spatial structure can introduce meaningful aspects of phylogenetic structure.

A naive technique was used to distill phylogenetic relationships from pedigree data.
Most Recent Common Ancestors (MRCA) were computed pairwise from the pedigree data to construct a distance matrix among extant individuals.
Unweighted Pair Group Method with Arithmetic mean (UPGMA) reconstruction based on this distance matrix yielded inferred phylogenetic structure \citep{sokal1958university}.
Finally, corrections to branch lengths were performed to accurately position the terminal nodes (i.e., extant organisms) at their known generational depths.

No effort was made to cluster extant organisms into taxa based on a relatedness threshold, so reconstructions contained non-informative branch structure among closely-related individuals.

\subsection{Hereditary Stratigraphic Column Recombination \& Gene-drive Mechanism}
\label{sec:gene-drive}

This mechanism only applied to species-level instrumentation and was used for the genealogical and population size inference experiments.

\subsection{Collision Probability Between Fixed Differentia Genes}
\label{sec:collision-probability}

Fixed genes' skew toward large-magnitude integers increases the probability of fixed-gene collision between two populations causing misdetection of shared ancestry between the populations.
Quantifying this collision probability effect is key to predicting the efficacy of this approach at scale.

Suppose independent populations of size $a$ and $b$.
The largest gene in each population will drive to fixation.
If each population members' gene is drawn from uniform distribution on integers $[0, u)$, then the probability of collision between the populations' fixed genes is

\begin{align*}
\frac{a \sum_{n=1}^{a + b - 1} \frac{u^{- n - 1} \left(\frac{u - 1}{u}\right)^{a + b - n - 1} \left(- \frac{{\binom{a - 1}{n}}}{{\binom{a + b - 1}{n}}} + 1\right) {\binom{a + b}{n + 1}}}{1 - \left(\frac{u - 1}{u}\right)^{a + b - 1}}}{a + b} \\
+ \frac{b \sum_{n=1}^{a + b - 1} \frac{u^{- n - 1} \left(\frac{u - 1}{u}\right)^{a + b - n - 1} \left(- \frac{{\binom{b - 1}{n}}}{{\binom{a + b - 1}{n}}} + 1\right) {\binom{a + b}{n + 1}}}{1 - \left(\frac{u - 1}{u}\right)^{a + b - 1}}}{a + b}.
\end{align*}

Derivation will be provided in supplementary materials.

For 32-bit differentia $u = 2^{32}$, collision occurs with $p < 0.5$ for two populations of size $a = b = 2 \times 2^{31}$ .
Collision occurs with $p < 0.01$ for two populations of size $a = b = 2^{26}$.
So, for 32-bit differentia populations with magnitude below a factor of 64 of the maximum representable differentia value, collision probability between is low.
Numerical considerations complicate evaluation of similar results for $u = 2^{64}$.

\subsection{Genealogical Inference Experiment}
\label{sec:genealogical-inference}

The experiment focuses on the evolution of populations under three different treatments: bag, ring, and allopatry. The populations, consisting of binary individuals, are evolved using a genetic algorithm provided by the DEAP Python library, where the fitness function is the sum of the elements in an individual. The experiment uses a hereditary stratigraphic column to track the lineage of organisms and records the history of the population. Each treatment was replicated 10 times and each instance of the experiment was saved separately as a YAML file.

The **bag treatment** involves a simple genetic algorithm, in which the entire population is involved in crossover (mating) and mutation, with no restrictions on mating partners. After selection, the new generation completely replaces the old one. The evolution runs for a predefined number of generations (NGEN).

In the **ring treatment**, the population is split into distinct subpopulations (islands), each of which evolves independently. The subpopulations are set up in a ring topology, and migration happens once per generation, allowing individuals from one subpopulation to replace individuals in another. The migration offset is randomly chosen. Evolution is run for NGEN generations.

The **allopatry treatment** represents a more complex evolutionary scenario. The population undergoes sympatric evolution for NGEN/2 generations, after which it is split into two, and each half evolves separately for NGEN/4 generations (two-island allopatry). Following this, one half of the population is split again into smaller islands, each evolving independently for NGEN/4 generations (three-island allopatry). This treatment allows the study of the evolutionary impact of geographical separation on populations.

TODO
Agglomerative trie-based reconstruction techniques developed in \citep{moreno2023toward}
Disagreement reconstructed phylogenies and corresponding distilled references was measured using quartet distance \citep{estabrook1985comparison}.

Talk about how meaningless extra information would distort conservatively.

\subsection{Population Size Inference Statistics}
\label{sec:population-size-inference-stats}

Consider a population composed of $n$ individuals, each holding a unique gene represented as an unsigned integer of a given precision.
At the outset, each gene value is assumed to be drawn from a uniform distribution across all possible values.
As generations evolve through sexual recombination, a "gene drive" mechanism is enforced where offspring inherit the larger gene value from their parents.
Over time, this mechanism results in the largest gene value becoming dominant or 'fixed' in the population.

An interesting property of this gene drive mechanism is the potential to estimate the original population size by observing the distribution of the fixed gene values.
By introducing a random variable $\mathbb{X}$ to represent the magnitude of the fixed gene value, we can explore its distribution. Assuming our gene value is uniformly distributed between 0 and 1, it is identified as $\beta(n, 1)$ \citep{gentle2009computational} distributed.
Hence, by observing the values of these fixed genes, it's possible to infer $n$, the original population size.
This approach is further extendable to cases involving multiple independent genes, each adhering to the same fixed precision and initial uniform distribution, providing a compelling estimation technique for population size.

Directly analogous techniques to estimate population size have also arisen in the context of decentralized, anonymous network engineering \citep{varagnolo2010distributed,hakan2012distributed}.
In these schemes, nodes independently draw a random vector of numerical values from a known distribution,
Values are exchanged through an aggregating function like (e.g., minimum, maximum, etc.), ultimately resulting in a consensus value fixing within the network.
Each node can then independently infer probabilistic information about the larger network, in a manner that is generally consistent across nodes.

Statistical details necessary to work with the fixed-gene-magnitude inference method follow, some of which are, to best knowledge, not yet reported.
Derivations are available at \url{https://github.com/mmore500/hereditary-stratigraph-concept/tree/master/binder/popsize}, and will be included in supplemental material.

\subsubsection{Maximum Likelihood Estimator}

The maximum likelihood estimator for population size given $i$ independent observations of fixed-gene magnitude $x_i$ is
\begin{align} \label{eqn:popsize_mle}
\hat{n}_\mathrm{mle} = -\frac{k}{\sum_{i=1}^k \log( x_i )}.
\end{align}

This estimator was also derived in \citep{varagnolo2010distributed}.

\subsubsection{Maximum Likelihood Estimator Mean Square Error}

Mean square error of the maximum likelihood estimator for population size given $k$ observations of fixed gene magnitude is

\begin{align*}
n^2 \frac{k^{2}+ k-2}{(k-1)^{2}(k-2)}.
\end{align*}

This result was also derived in \citep{varagnolo2010distributed}.

\subsubsection{Maximum Likelihood Estimator Expected Value}

Expected value for the maximum likelihood population size estimator $\hat{n}_\mathrm{mle}$ is given as

\begin{align*}
E(\hat{n}_\mathrm{mle})
= n\frac{k}{k-1}
\end{align*}

for $k>1$ \citep{varagnolo2010distributed}.
This value can be subtracted from the maximum likelihood estimator to yield a mean-unbiased population size estimator.

\subsubsection{Confidence Interval}

Computation of confidence intervals is necessary of facilitate experimental inference from population size estimators.
Formulations derived from the maximum likelihood estimator are provided below.

For a single observation of fixed gene magnitude $\hat{x}$, the population size $n$ can be estimated with $c\%$ confidence to fall within the interval

\begin{align*}
\Big(
\frac{\log(\frac{1+c/100}{2})}{\log\hat{x}},
\frac{\log(\frac{1-c/100}{2})}{\log\hat{x}}
\Big).
\end{align*}

The 95\% confidence interval spans a 145-fold order of magnitude and a 99\% confidence interval spans a 1057-fold order of magnitude.

For $k$ observations of fixed gene magnitude $\hat{x}_i$, the population size $n$ can be estimated with $c\%$ confidence to fall within the interval $(\hat{n}_\mathrm{lb}, \hat{n}_\mathrm{ub})$ where $\hat{n}_\mathrm{lb}$ is the solution to

\begin{align} \label{eqn:popsize_mle_ci_lb}
0
&= 2\Gamma(k, -\hat{n}_\mathrm{lb}\log(\prod_{i=1}^k\hat{x}_i)) - (c/100+1)\Gamma(k)
\end{align}

and $\hat{n}_\mathrm{ub}$ is the solution to

\begin{align} \label{eqn:popsize_mle_ci_ub}
  0
  &= 2\Gamma(k, -\hat{n}_\mathrm{ub}\log(\prod_{i=1}^k\hat{x}_i))) - \Gamma(k)(1-c/100).
\end{align}

Four independent observations are sufficient to provide a 95\% confidence interval spanning 8-fold magnitude or a 99\% confidence interval spanning a 16-fold magnitude.
Nine independent observations are sufficient for a 95\% confidence interval spanning a factor spanning 4-fold magnitude or a 99\% confidence interval spanning a factor of 6-fold magnitude.
33 independent observations are sufficient for a 95\% confidence interval spanning 2-fold magnitude or a 99\% confidence interval spanning 2.5-fold magnitude.

Note that the width of the confidence interval will scale as a constant factor of as $n \to \infty$.

\subsubsection{Median-unbiased Estimator}

A median-unbiased estimator $\hat{n}_\mathrm{mle}$ for population can be trivially derived from the confidence interval, as the numerical solution of

\begin{align*}
0
&= 2\Gamma(k, -\hat{n}_\mathrm{mumle}\log(\prod_{i=1}^k x_i)) - \Gamma(k).
\end{align*}

\subsubsection{Credible Intervals}

Computation of credible intervals facilitates bayesian experimental inference from population size estimators.
Derivation assumes a uniform prior over population size.
The credibility contained within a factor of the maximum likelihood estimate $\hat{n}_\mathrm{mle}$ can be calculated as


\begin{align*}
\frac{- \gamma(k + 1, \frac{k}{f}) + \gamma(k + 1, f k)}{k!}.
\end{align*}

By its form, the credibility contained within a factor of maximum likelihood estimate remains constant across population sizes $n$.
Credibility intervals require similar sample sizes to provide $n$-fold magnitude spans as those for confidence intervals discussed above.

\subsubsection{Rolling Population Size Estimation} \label{sec:rolling_estimation}

Experiments reported here use a simple rolling process to aggregate ten preceding fixed-gene magnitudes to compute a population size estimate and confidence intervals.
More sophisticated regularizations have been proposed to consolidate time series estimates of dynamically-changing network sizes \citep{hakan2012distributed}.

\subsection{Example Population Size Inference}
\label{sec:ne-process-example}

\begin{figure}
  \centering

  \begin{minipage}{.75\textwidth}
    \centering
    \includegraphics[height=0.3\textheight]{notebooks/notebooks/teeplots/notebook=ne-inference+replicate=0+treatment=control+viz=scatterplot-differentia-magnitude+ext=}
  \end{minipage}%
  \begin{minipage}{.2\textwidth}
    \subcaption{Differential Magnitude Scatterplot}
    \label{fig:ne-process-example:differentia}
  \end{minipage}

  \vspace{1em}

  \begin{minipage}{.75\textwidth}
    \centering
    \includegraphics[height=0.3\textheight]{notebooks/notebooks/teeplots/notebook=ne-inference+replicate=0+treatment=control+viz=scatterplot-popsize-estimates+ext=}
  \end{minipage}%
  \begin{minipage}{.2\textwidth}
    \subcaption{Population Size Estimates Scatterplot}
    \label{fig:ne-process-example:singleton-est}
  \end{minipage}

  \vspace{1em}

  \begin{minipage}{.75\textwidth}
    \centering
    \includegraphics[height=0.3\textheight]{notebooks/notebooks/teeplots/notebook=ne-inference+replicate=0+treatment=control+viz=plot-running-estimation+x=rank+y=population-size+ext=}
  \end{minipage}%
  \begin{minipage}{.2\textwidth}
    \subcaption{Running Estimation}
    \label{fig:ne-process-example:running-est}
  \end{minipage}

  \caption{TODO}
  \label{fig:ne-process-example}
\end{figure}


% notebooks/notebooks/teeplots/notebook=ne-inference+replicate=0+treatment=control+viz=plot-running-estimation+x=rank+y=population-size+ext=.pdf
%
% notebooks/notebooks/teeplots/notebook=ne-inference+replicate=0+treatment=control+viz=scatterplot-differentia-magnitude+ext=.pdf
%
% notebooks/notebooks/teeplots/notebook=ne-inference+replicate=0+treatment=control+viz=scatterplot-popsize-estimates+ext=.pdf


% notebooks/notebooks/teeplots/notebook=ne-inference+replicate=0+treatment=bottleneck+viz=plot-running-estimation+x=rank+y=population-size+ext=.pdf
%
% notebooks/notebooks/teeplots/notebook=ne-inference+replicate=0+treatment=bottleneck+viz=scatterplot-differentia-magnitude+ext=.pdf
%
% notebooks/notebooks/teeplots/notebook=ne-inference+replicate=0+treatment=bottleneck+viz=scatterplot-popsize-estimates+ext=.pdf

% notebooks/notebooks/teeplots/notebook=ne-inference+replicate=0+treatment=range-expansion+viz=plot-running-estimation+x=rank+y=population-size+ext=.pdf
%
% notebooks/notebooks/teeplots/notebook=ne-inference+replicate=0+treatment=range-expansion+viz=scatterplot-differentia-magnitude+ext=.pdf
%
% notebooks/notebooks/teeplots/notebook=ne-inference+replicate=0+treatment=range-expansion+viz=scatterplot-differentia-magnitude+ext=.pdf
%
% notebooks/notebooks/teeplots/notebook=ne-inference+replicate=0+treatment=selection-pressure+viz=plot-running-estimation+x=rank+y=population-size+ext=.pdf
%
% notebooks/notebooks/teeplots/notebook=ne-inference+replicate=0+treatment=selection-pressure+viz=scatterplot-differentia-magnitude+ext=.pdf
%
% notebooks/notebooks/teeplots/notebook=ne-inference+replicate=0+treatment=selection-pressure+viz=scatterplot-popsize-estimates+ext=.pdf


Figure \ref{fig:ne-process-example} shows an example of population size inference over the course of an experiment.

Panel \ref{fig:ne-process-example:differentia} shows magnitudes of fixed differentia across the generational record extracted from a sample specimen at the end of the run.

Population size estimates can be computed at each generation using the maximum likelihood estimator (Equation \ref{eqn:popsize_mle}, as shown in Panel \ref{fig:ne-process-example:singleton-est}.
The true population size is annotated as a horizontal blue line.

To generate more robust inference, inference was pooled over rolling sets of 10 fixed differentia.
Population size estimates were again performed using Equation \ref{eqn:popsize_mle} with confidence interval bounds computed from Equations \ref{eqn:popsize_mle_ci_lb} and \ref{eqn:popsize_mle_ci_ub}.
Figure \ref{fig:ne-process-example:running-est} plots this running estimation.
Note that some discrepancy is expected between absolute population size (horizontal line) and effective population size $N_e$ (estimated) due to demographic factors.

\subsection{Population Size Inference Experiment}
\label{sec:population-size-inference-experiments}

The experiment tested the ability of the population size estimation process (Section \ref{sec:ne-process-example} to detect effective population size differences between populations and effective population size changes within a population over time.
Four different treatment conditions were compared: bottleneck, range expansion, selection pressure, and control.
Each treatment was replicated 10 times to account for stochastic variability in evolutionary processes.

The bottleneck treatment simulated a population reduction event.
The population size was kept at 100 for 67 generations, reduced an order of magnitude to 10 for 66 generations, and then returned to 100 for another 67 generations.

The range expansion treatment simulated gradual population size expansion.
The population size was initiated at at 10 for 67 generations, then increased linearly for 66 generations to 142 at generation 133, and then maintained at 142 for another 67 generations.

The selection-pressure treatment modified the selection intensity during the evolutionary process.
This reduced the effective population size by increasing the number of population members eliminated without contributing to the next generation. gene pool.
High selection pressure was applied for 67 generations (tournament size 8). Then, selection pressure was eliminated for 66 generations (tournament size of 1).
Finally, high selection pressure (tournament size 8) was reinstated for the last 67 generations.

The control treatment was run with a constant population size of 100 for 200 generations.

\subsection{Distributed Gene Prevalence Estimation}
\label{sec:dist-gene-prevalence-est}

\subsection{Positive Selection Inference Experiment}
\label{sec:positive-selection-inference-experiment}

The fitness advantage task models the introduction of an atomic gene with tunable fitness advantage into an otherwise neutral background.
Each individual in the population was represented by a single floating-point number, representing the focal gene.
Gene values were restricted between 0.0 and 1.0.
Fitness score was defined as the sum of its own value and a random number drawn from a continuous, uniform unit-valued distribution.
So, individuals' gene value corresponded directly to probabilistic fitness advantage.
For example, a value of 0.2 would give an average 20\% selective advantage.
Fitness scores for each individual were calculated once per generation and used for all tournaments.

All individuals were initialized with gene value 0.0.
At generation 50, one organism's gene value was set to either 0.0\footnote{The smallest representable positive value was set for fitness advantage treatment 0.0 so the introduced gene could be differentiated from the background gene. Infinitesimal small value was used so as to have no meaningfully detectable effect on selection.}, 0.1, or 1.0.
This operation was repeated at subsequent generations if the introduced gene value went extinct.
This procedure enabled comparison of a strong selective sweep for the gene (fitness advantage 1.0), a weaker selective sweep for the gene (fitness advantage 0.1) and a control treatment where no fitness advantage was introduced and pure drift dynamics were at play (fitness advantage 0.0).
Underlying selective sweep dynamics were measured by recording gene copy count at each generation .

Synchronous selection with tournament size 2 was performed each generation.
200 generations were simulated, with a constant population size of 400.
All parents for the upcoming population were selected before turnover of the entire population.
The crossover mating operator selected a gene value from among the two parents' genes with equal probability.
No mutation was applied.

\subsection{Software \& Data}
\label{sec:software-data}

Software, data, and analyses used in this work are hosted on GitHub at \url{github.com/mmore500/hstrat-recomb-concept/tree/notebooks/notebooks}.

Data structures and algorithms associated with hereditary stratigraphy methodology are published in the hstrat Python package, made available on PyPi and on GitHub at \url{github.com/mmore500/hstrat} under the MIT license \citep{moreno2022hstrat}.
Recombination features developed for this project, as well as corresponding C++ implementations of hereditary stratigraphy data structures and algorithms, are on the project's near-term roadmap.

This project benefited greatly from open source scientific software, including BioPython \citep{cock2009biopython}, Distributed Evolutionary Algorithms in Python (DEAP) \citep{fortin2012deap}, Matplotlib \citep{hunter2007matplotlib}, NetworkX \citep{hagberg2008networkx}, Jupyter \citep{loizides2016jupyter}, pandas \citep{reback2020pandas,mckinney2010pandas}, SciPy \citep{pauli2020scipy}, seaborn \citep{waskom2021seaborn}, SymPy \citep{meurer2017sympy}, and tqdist \citep{sand2014tqdist}.
The Artificial Life data standard for phylogenies facilitated tool interoperation \citep{lalejini2019data}.
