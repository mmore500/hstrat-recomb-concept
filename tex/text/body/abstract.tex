\abstract{
The structure of relatedness among members of an evolved population tells much of its evolutionary history.
In application-oriented evolutionary computation (EC), such phylogenetic information can guide algorithm selection and tuning.
Although traditional direct tracking approaches provide the perfect phylogenetic record, sexual recombination complicates management and analysis of this data.
Taking inspiration from biological science, this work explores a reconstruction-based approach that uses end-state genetic information to estimate phylogenetic history after the fact.
We apply recently-developed ``hereditary stratigraphy'' genome annotations to lineages with sexual recombination to design devices germane to species phylogenies and gene trees.
As shown through a series of validation experiments, proposed instrumentation can discern genealogical history, population size changes, and selective sweeps.
Fully decentralized by nature, these methods afford new observability at scale, in particular, for distributed EC systems.
Such capabilities anticipate continued growth of computational resources available to EC.
Accompanying open source software aims to expedite application of reconstruction-based phylogenetic analysis where pertinent.
}
