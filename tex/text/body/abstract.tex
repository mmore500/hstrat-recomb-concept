\abstract{
The structure of relatedness among members of an evolved population tells much of its evolutionary history.
In application-oriented evolutionary computation (EC), such phylogeneetic information can guide algorithm selection and tuning.
Although traditional direct tracking approaches yield perfect phylogenetic records, sexual recombination complicates data management and analysis under this paradigm.
Taking inspiration from biological science, this work explores a reconstruction-based approach that uses end-state genetic information to estimate phylogenetic history after the fact.
We apply recently-developed ``hereditary stratigraphy'' genome annotations to sexual lineages, suggesting devices germane to species phylogenies and gene trees.
Proposed instrumentation can discern genealogical history, population size changes, and selective sweeps, all shown through a series of validation experiments.
Fully decentralized by nature, these methods afford new observational scalability, in particular, to distributed EC systems.
Such capabilities anticipate continued growth of computational resources available to EC.
Accompanying open source software aims to expedite incorporation in practice where pertinent.
}
