\section{Supplemental Material}

\subsection{One-Max Task}
\label{sec:one-max}

We used the one-max task for population size and genealogical inference experiments.
This task selects over binary string of length 100 for the number of 1's set.
Fitness preferred individuals with more 1's over individuals with fewer.
Population size was 200, initialized randomly.
In treatments where population size decreased during an evolutionary run, excess individuals were designated for elimination randomly.
In treatments where population size increased during an evolutionary run,
new population slots were filled through selection over the existing population.

Selection was performed using tournament selection with synchronous generations.
Default tournament size was 2.
Evolutionary runs lasted 100 generations.
Two-point crossover (mating) and bit-flip mutation with per-bit probability 0.05 each generation prior to selection.
Operator choice and parameter selections follows one-max example code from the DEAP package \citep{fortin2012deap}.

\subsection{Population Size Inference Example}
\label{sec:population-size-inference-example}

\begin{figure}
  \centering

  \begin{minipage}{.75\textwidth}
    \centering
    \includegraphics[height=0.3\textheight]{notebooks/notebooks/teeplots/notebook=ne-inference+replicate=0+treatment=control+viz=scatterplot-differentia-magnitude+ext=}
  \end{minipage}%
  \begin{minipage}{.2\textwidth}
    \subcaption{Differential Magnitude Scatterplot}
    \label{fig:ne-process-example:differentia}
  \end{minipage}

  \vspace{1em}

  \begin{minipage}{.75\textwidth}
    \centering
    \includegraphics[height=0.3\textheight]{notebooks/notebooks/teeplots/notebook=ne-inference+replicate=0+treatment=control+viz=scatterplot-popsize-estimates+ext=}
  \end{minipage}%
  \begin{minipage}{.2\textwidth}
    \subcaption{Population Size Estimates Scatterplot}
    \label{fig:ne-process-example:singleton-est}
  \end{minipage}

  \vspace{1em}

  \begin{minipage}{.75\textwidth}
    \centering
    \includegraphics[height=0.3\textheight]{notebooks/notebooks/teeplots/notebook=ne-inference+replicate=0+treatment=control+viz=plot-running-estimation+x=rank+y=population-size+ext=}
  \end{minipage}%
  \begin{minipage}{.2\textwidth}
    \subcaption{Running Estimation}
    \label{fig:ne-process-example:running-est}
  \end{minipage}

  \caption{TODO}
  \label{fig:ne-process-example}
\end{figure}


% notebooks/notebooks/teeplots/notebook=ne-inference+replicate=0+treatment=control+viz=plot-running-estimation+x=rank+y=population-size+ext=.pdf
%
% notebooks/notebooks/teeplots/notebook=ne-inference+replicate=0+treatment=control+viz=scatterplot-differentia-magnitude+ext=.pdf
%
% notebooks/notebooks/teeplots/notebook=ne-inference+replicate=0+treatment=control+viz=scatterplot-popsize-estimates+ext=.pdf


% notebooks/notebooks/teeplots/notebook=ne-inference+replicate=0+treatment=bottleneck+viz=plot-running-estimation+x=rank+y=population-size+ext=.pdf
%
% notebooks/notebooks/teeplots/notebook=ne-inference+replicate=0+treatment=bottleneck+viz=scatterplot-differentia-magnitude+ext=.pdf
%
% notebooks/notebooks/teeplots/notebook=ne-inference+replicate=0+treatment=bottleneck+viz=scatterplot-popsize-estimates+ext=.pdf

% notebooks/notebooks/teeplots/notebook=ne-inference+replicate=0+treatment=range-expansion+viz=plot-running-estimation+x=rank+y=population-size+ext=.pdf
%
% notebooks/notebooks/teeplots/notebook=ne-inference+replicate=0+treatment=range-expansion+viz=scatterplot-differentia-magnitude+ext=.pdf
%
% notebooks/notebooks/teeplots/notebook=ne-inference+replicate=0+treatment=range-expansion+viz=scatterplot-differentia-magnitude+ext=.pdf
%
% notebooks/notebooks/teeplots/notebook=ne-inference+replicate=0+treatment=selection-pressure+viz=plot-running-estimation+x=rank+y=population-size+ext=.pdf
%
% notebooks/notebooks/teeplots/notebook=ne-inference+replicate=0+treatment=selection-pressure+viz=scatterplot-differentia-magnitude+ext=.pdf
%
% notebooks/notebooks/teeplots/notebook=ne-inference+replicate=0+treatment=selection-pressure+viz=scatterplot-popsize-estimates+ext=.pdf


Panel \ref{fig:ne-process-example:differentia} shows magnitudes of fixed differentia across the generational record extracted from a sample specimen at the end of the run.

Population size estimates can be computed at each generation using the maximum likelihood estimator (given in Equation \ref{eqn:popsize_mle}), as shown in Panel \ref{fig:ne-process-example:singleton-est}.
The true population size is annotated as a horizontal blue line.

To generate more robust inference, inference was pooled over rolling sets of 10 fixed differentia.
Population size estimates were again performed using the maximum likelihood estimator given in Equation \ref{eqn:popsize_mle} with confidence interval bounds computed via Equation \ref{eqn:popsize_mle_ci}.
Supplemental Figure \ref{fig:ne-process-example:running-est} plots this running estimation.
Note that some discrepancy is expected between absolute population size (horizontal line) and effective population size $N_e$ (estimated) due to demographic factors.



\subsection{Supplementary Figures}

\begin{figure}
  \centering
  \begin{subfigure}[b]{0.6\textwidth}
    \centering
    \includegraphics[width=\textwidth]{notebooks/notebooks/teeplots/viz=beta-pdf+ext=}
    % \caption{empty}
    % \label{fig:empty}
  \end{subfigure}%
  \begin{subfigure}[b]{0.4\textwidth}
    \centering
    \includegraphics[width=\textwidth]{img/dice-pool}
    \vspace{3ex}
    % \caption{empty}
    % \label{fig:empty}
  \end{subfigure}
  \caption{
    Working principle of population size estimation.
    Increasing population size skews probability density function for population maximum value of generated fingerprints (referred to above as ``differentia'').
  }
  \label{fig:beta-explain}
\end{figure}


\begin{figure}
  \centering
  \includegraphics[width=0.8\textwidth]{notebooks/notebooks/teeplots/viz=boxplot-quartet+x=treatment+y=quartet-distance+ext=}
  \caption{
    Distribution of normalized quartet distances between reconstructed phylogenies and references distilled from tracked pedigree.
    Lower indicates less reconstruction error.
    Notches represent bootstrapped 95\% confidence interval.
    Horizontal blue line indicates expected quartet distance between random trees.
    Note that some reconstruction error is expected, especially in control treatment, due to inclusion of effectively arbitrary phylogenetic structure among well-mixed population components (Section \ref{sec:phylogeny-extraction}).
  }
  \label{fig:species-reconstruction-error}
\end{figure}


\begin{sidewaysfigure}
  \centering
  \begin{minipage}{.8\textwidth} % adjust the width as needed

    \begin{minipage}{0.5\textwidth}
      \centering
      \includegraphics[height=0.3\textheight]{notebooks/notebooks/teeplots/notebook=ne-inference+replicate=0+treatment=bottleneck+viz=plot-running-estimation+x=rank+y=population-size+ext=}
      \subcaption{Bottleneck treatment}
      \label{fig:ne-example-replicates:bottleneck}
    \end{minipage}%
    \begin{minipage}{0.5\textwidth}
      \centering
      \includegraphics[height=0.3\textheight]{notebooks/notebooks/teeplots/notebook=ne-inference+replicate=0+treatment=control+viz=plot-running-estimation+x=rank+y=population-size+ext=}
      \subcaption{Control treatment}
      \label{fig:ne-example-replicates:control}
    \end{minipage}

    \vspace{1cm}

    \begin{minipage}{0.5\textwidth}
      \centering
      \includegraphics[height=0.3\textheight]{notebooks/notebooks/teeplots/notebook=ne-inference+replicate=0+treatment=range-expansion+viz=plot-running-estimation+x=rank+y=population-size+ext=}
      \subcaption{Range expansion treatment}
      \label{fig:ne-example-replicates:range_expansion}
    \end{minipage}%
    \begin{minipage}{0.5\textwidth}
      \centering
      \includegraphics[height=0.3\textheight]{notebooks/notebooks/teeplots/notebook=ne-inference+replicate=0+treatment=selection-pressure+viz=plot-running-estimation+x=rank+y=population-size+ext=}
      \subcaption{Selection pressure treatment}
      \label{fig:ne-example-replicates:selection_pressure}
    \end{minipage}

  \end{minipage}
  \hfill % Creates horizontal space. Can also use \hspace{<len>}
  \begin{minipage}{.15\textwidth} % adjust the width as needed
    \caption{Comparison of population size estimation}
    \label{fig:ne-example-replicates}
  \end{minipage}

\end{sidewaysfigure}


% notebooks/notebooks/teeplots/notebook=ne-inference+replicate=0+treatment=bottleneck+viz=plot-running-estimation+x=rank+y=population-size+ext=.pdf
%
% notebooks/notebooks/teeplots/notebook=ne-inference+replicate=0+treatment=control+viz=plot-running-estimation+x=rank+y=population-size+ext=.pdf
%
% notebooks/notebooks/teeplots/notebook=ne-inference+replicate=0+treatment=range-expansion+viz=plot-running-estimation+x=rank+y=population-size+ext=.pdf
%
% notebooks/notebooks/teeplots/notebook=ne-inference+replicate=0+treatment=selection-pressure+viz=plot-running-estimation+x=rank+y=population-size+ext=.pdf


\begin{figure}
  \centering

  \begin{subfigure}{\textwidth}
    \begin{minipage}{0.7\textwidth}
      \includegraphics[width=\linewidth]{notebooks/notebooks/teeplots/hue=mann-whitney-significant-at-alpha-0-01+viz=facet-heatmap+x=generation-b+y=generation-a+ext=}
    \end{minipage}%
    \begin{minipage}{0.25\textwidth}
      \caption{Mann-Whitney comparison of 33 annotations at time points surrounding each target generation, with significance threshold $\alpha = 0.01$.}
      \label{fig:ne-detection-matrix:mann-whitney}
    \end{minipage}
  \end{subfigure}

  \vspace{1em} % adjust the vertical spacing between the subfigures

  \begin{subfigure}{\textwidth}
    \begin{minipage}{0.7\textwidth}
      \includegraphics[width=\linewidth]{notebooks/notebooks/teeplots/hue=nonoverlapping-ci+viz=facet-heatmap+x=generation-b+y=generation-a+ext=}
    \end{minipage}%
    \begin{minipage}{0.25\textwidth}
      \caption{Non-overlapping MLE 95\% confidence intervals for rolling 10-sample estimate at target generations.}
      \label{fig:ne-detection-matrix:ci}
    \end{minipage}
  \end{subfigure}

  \caption{Counts of replicates where significant differences in effective population size ($N_e$) were detected between time point pairs.
  Counts are out of 10 total replicates attempted.
  All time point pairs had true differences in $N_e$, except same-time point pairs and time point pairs in the control experiment (Section \ref{sec:population-size-inference-experiments}.
  }
  \label{fig:ne-detection-matrix}
\end{figure}

% notebooks/notebooks/teeplots/hue=mann-whitney-significant-at-alpha-0-01+viz=facet-heatmap+x=generation-b+y=generation-a+ext=.pdf
%
% notebooks/notebooks/teeplots/hue=nonoverlapping-ci+viz=facet-heatmap+x=generation-b+y=generation-a+ext=.pdf


\begin{figure}
  \centering
  \includegraphics[width=0.8\textwidth]{notebooks/notebooks/teeplots/hue=fitness-advantage+viz=lineplot-detection+x=threshold+y=replicate-count+ext=}
  \caption{
    Gene selection detection rates across detection thresholds for each fitness advantage level among 10 replicates.
    Fitness advantage 0.0 inferred no selective benefit, so all selection detections on this treatment are false positives.
    Fitness advantage 0.1 experienced relatively weak selection and fitness advantage 1.0 experienced strong selection.
    Detection threshold 200 distinguishes treatment 0.0 and 0.1 with one false positive and one false negative.
    Fitness advantage 1.0 has all replicates detected across all shown threshold values.
    Selection is detected for a replicate if any 16-generation rolling sum of gene prevalence annotation bit count (Section \ref{sec:dist-gene-prevalence-est}) exceeds the threshold.
  }
  \label{fig:selection-sensitivity-specificity}
\end{figure}


% \input{text/supplement/example.tex}
